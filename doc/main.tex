% !TeX spellcheck = en_US
\documentclass[a4paper,12pt]{article}
\usepackage{amsmath}
\usepackage{amsfonts}
\usepackage{amssymb}
\usepackage{bbold}
\usepackage{cancel}
\author{Moritz Schubotz}
\title{Technical details on texvc identifiere extraction}
\begin{document}
\maketitle
\section{Introduction}
This document describes which mathematical symbols are identified as identifiers.
In general every single Latin letter [a-zA-Z] is regarded as identifier.
In addition, we accept multi-letter-subscripts that match [0-9a-zA-Z]+, such as $a_0$ but also $\varepsilon_{ijk}$.
Moreover, the Literals described in section \ref{sc.lit}, and the Identifier variants (section \ref{sc.var}) are supported.
\section{Literals}\label{sc.lit}
The following literals are supported:

\texttt{\textbackslash aleph} is rendered as $\aleph$


\texttt{\textbackslash alpha} is rendered as $\alpha$


\texttt{\textbackslash amalg} is rendered as $\amalg$


\texttt{\textbackslash And} is rendered as $\And$


\texttt{\textbackslash angle} is rendered as $\angle$


\texttt{\textbackslash ast} is rendered as $\ast$


\texttt{\textbackslash asymp} is rendered as $\asymp$


\texttt{\textbackslash backepsilon} is rendered as $\backepsilon$


\texttt{\textbackslash backprime} is rendered as $\backprime$


\texttt{\textbackslash backsim} is rendered as $\backsim$


\texttt{\textbackslash backsimeq} is rendered as $\backsimeq$


\texttt{\textbackslash barwedge} is rendered as $\barwedge$


\texttt{\textbackslash Bbbk} is rendered as $\Bbbk$


\texttt{\textbackslash because} is rendered as $\because$


\texttt{\textbackslash beta} is rendered as $\beta$


\texttt{\textbackslash beth} is rendered as $\beth$


\texttt{\textbackslash between} is rendered as $\between$


\texttt{\textbackslash bigcap} is rendered as $\bigcap$


\texttt{\textbackslash bigcirc} is rendered as $\bigcirc$


\texttt{\textbackslash bigcup} is rendered as $\bigcup$


\texttt{\textbackslash bigodot} is rendered as $\bigodot$


\texttt{\textbackslash bigoplus} is rendered as $\bigoplus$


\texttt{\textbackslash bigotimes} is rendered as $\bigotimes$


\texttt{\textbackslash bigsqcup} is rendered as $\bigsqcup$


\texttt{\textbackslash bigstar} is rendered as $\bigstar$


\texttt{\textbackslash bigtriangledown} is rendered as $\bigtriangledown$


\texttt{\textbackslash bigtriangleup} is rendered as $\bigtriangleup$


\texttt{\textbackslash biguplus} is rendered as $\biguplus$


\texttt{\textbackslash bigvee} is rendered as $\bigvee$


\texttt{\textbackslash bigwedge} is rendered as $\bigwedge$


\texttt{\textbackslash blacklozenge} is rendered as $\blacklozenge$


\texttt{\textbackslash blacksquare} is rendered as $\blacksquare$


\texttt{\textbackslash blacktriangle} is rendered as $\blacktriangle$


\texttt{\textbackslash blacktriangledown} is rendered as $\blacktriangledown$


\texttt{\textbackslash blacktriangleleft} is rendered as $\blacktriangleleft$


\texttt{\textbackslash blacktriangleright} is rendered as $\blacktriangleright$


\texttt{\textbackslash bot} is rendered as $\bot$


\texttt{\textbackslash bowtie} is rendered as $\bowtie$


\texttt{\textbackslash Box} is rendered as $\Box$


\texttt{\textbackslash boxdot} is rendered as $\boxdot$


\texttt{\textbackslash boxminus} is rendered as $\boxminus$


\texttt{\textbackslash boxplus} is rendered as $\boxplus$


\texttt{\textbackslash boxtimes} is rendered as $\boxtimes$


\texttt{\textbackslash bullet} is rendered as $\bullet$


\texttt{\textbackslash bumpeq} is rendered as $\bumpeq$


\texttt{\textbackslash Bumpeq} is rendered as $\Bumpeq$


\texttt{\textbackslash cap} is rendered as $\cap$


\texttt{\textbackslash Cap} is rendered as $\Cap$


\texttt{\textbackslash cdot} is rendered as $\cdot$


\texttt{\textbackslash cdots} is rendered as $\cdots$


\texttt{\textbackslash centerdot} is rendered as $\centerdot$


\texttt{\textbackslash checkmark} is rendered as $\checkmark$


\texttt{\textbackslash chi} is rendered as $\chi$


\texttt{\textbackslash circ} is rendered as $\circ$


\texttt{\textbackslash circeq} is rendered as $\circeq$


\texttt{\textbackslash circlearrowleft} is rendered as $\circlearrowleft$


\texttt{\textbackslash circlearrowright} is rendered as $\circlearrowright$


\texttt{\textbackslash circledast} is rendered as $\circledast$


\texttt{\textbackslash circledcirc} is rendered as $\circledcirc$


\texttt{\textbackslash circleddash} is rendered as $\circleddash$


\texttt{\textbackslash circledS} is rendered as $\circledS$


\texttt{\textbackslash clubsuit} is rendered as $\clubsuit$


\texttt{\textbackslash colon} is rendered as $\colon$


\texttt{\textbackslash complement} is rendered as $\complement$


\texttt{\textbackslash cong} is rendered as $\cong$


\texttt{\textbackslash coprod} is rendered as $\coprod$


\texttt{\textbackslash cup} is rendered as $\cup$


\texttt{\textbackslash Cup} is rendered as $\Cup$


\texttt{\textbackslash curlyeqprec} is rendered as $\curlyeqprec$


\texttt{\textbackslash curlyeqsucc} is rendered as $\curlyeqsucc$


\texttt{\textbackslash curlyvee} is rendered as $\curlyvee$


\texttt{\textbackslash curlywedge} is rendered as $\curlywedge$


\texttt{\textbackslash curvearrowleft} is rendered as $\curvearrowleft$


\texttt{\textbackslash curvearrowright} is rendered as $\curvearrowright$


\texttt{\textbackslash dagger} is rendered as $\dagger$


\texttt{\textbackslash daleth} is rendered as $\daleth$


\texttt{\textbackslash dashv} is rendered as $\dashv$


\texttt{\textbackslash ddagger} is rendered as $\ddagger$


\texttt{\textbackslash ddots} is rendered as $\ddots$


\texttt{\textbackslash delta} is rendered as $\delta$


\texttt{\textbackslash Delta} is rendered as $\Delta$


\texttt{\textbackslash diagdown} is rendered as $\diagdown$


\texttt{\textbackslash diagup} is rendered as $\diagup$


\texttt{\textbackslash diamond} is rendered as $\diamond$


\texttt{\textbackslash Diamond} is rendered as $\Diamond$


\texttt{\textbackslash diamondsuit} is rendered as $\diamondsuit$


\texttt{\textbackslash digamma} is rendered as $\digamma$


\texttt{\textbackslash displaystyle} is rendered as $\displaystyle$


\texttt{\textbackslash divideontimes} is rendered as $\divideontimes$


\texttt{\textbackslash doteq} is rendered as $\doteq$


\texttt{\textbackslash doteqdot} is rendered as $\doteqdot$


\texttt{\textbackslash dotplus} is rendered as $\dotplus$


\texttt{\textbackslash dots} is rendered as $\dots$


\texttt{\textbackslash dotsb} is rendered as $\dotsb$


\texttt{\textbackslash dotsc} is rendered as $\dotsc$


\texttt{\textbackslash dotsi} is rendered as $\dotsi$


\texttt{\textbackslash dotsm} is rendered as $\dotsm$


\texttt{\textbackslash dotso} is rendered as $\dotso$


\texttt{\textbackslash doublebarwedge} is rendered as $\doublebarwedge$


\texttt{\textbackslash downdownarrows} is rendered as $\downdownarrows$


\texttt{\textbackslash downharpoonleft} is rendered as $\downharpoonleft$


\texttt{\textbackslash downharpoonright} is rendered as $\downharpoonright$


\texttt{\textbackslash ell} is rendered as $\ell$


\texttt{\textbackslash emptyset} is rendered as $\emptyset$


\texttt{\textbackslash epsilon} is rendered as $\epsilon$


\texttt{\textbackslash eqcirc} is rendered as $\eqcirc$


\texttt{\textbackslash eqsim} is rendered as $\eqsim$


\texttt{\textbackslash eqslantgtr} is rendered as $\eqslantgtr$


\texttt{\textbackslash eqslantless} is rendered as $\eqslantless$


\texttt{\textbackslash eta} is rendered as $\eta$


\texttt{\textbackslash eth} is rendered as $\eth$


\texttt{\textbackslash exists} is rendered as $\exists$


\texttt{\textbackslash fallingdotseq} is rendered as $\fallingdotseq$


\texttt{\textbackslash Finv} is rendered as $\Finv$


\texttt{\textbackslash flat} is rendered as $\flat$


\texttt{\textbackslash frown} is rendered as $\frown$


\texttt{\textbackslash Game} is rendered as $\Game$


\texttt{\textbackslash gamma} is rendered as $\gamma$


\texttt{\textbackslash Gamma} is rendered as $\Gamma$


\texttt{\textbackslash geq} is rendered as $\geq$


\texttt{\textbackslash geqq} is rendered as $\geqq$


\texttt{\textbackslash geqslant} is rendered as $\geqslant$


\texttt{\textbackslash gets} is rendered as $\gets$


\texttt{\textbackslash gg} is rendered as $\gg$


\texttt{\textbackslash ggg} is rendered as $\ggg$


\texttt{\textbackslash gimel} is rendered as $\gimel$


\texttt{\textbackslash gnapprox} is rendered as $\gnapprox$


\texttt{\textbackslash gneq} is rendered as $\gneq$


\texttt{\textbackslash gneqq} is rendered as $\gneqq$


\texttt{\textbackslash gnsim} is rendered as $\gnsim$


\texttt{\textbackslash gtrapprox} is rendered as $\gtrapprox$


\texttt{\textbackslash gtrdot} is rendered as $\gtrdot$


\texttt{\textbackslash gtreqless} is rendered as $\gtreqless$


\texttt{\textbackslash gtreqqless} is rendered as $\gtreqqless$


\texttt{\textbackslash gtrless} is rendered as $\gtrless$


\texttt{\textbackslash gtrsim} is rendered as $\gtrsim$


\texttt{\textbackslash gvertneqq} is rendered as $\gvertneqq$


\texttt{\textbackslash hbar} is rendered as $\hbar$


\texttt{\textbackslash heartsuit} is rendered as $\heartsuit$


\texttt{\textbackslash hookleftarrow} is rendered as $\hookleftarrow$


\texttt{\textbackslash hookrightarrow} is rendered as $\hookrightarrow$


\texttt{\textbackslash hslash} is rendered as $\hslash$


\texttt{\textbackslash Im} is rendered as $\Im$


\texttt{\textbackslash imath} is rendered as $\imath$


\texttt{\textbackslash implies} is rendered as $\implies$


\texttt{\textbackslash infty} is rendered as $\infty$


\texttt{\textbackslash injlim} is rendered as $\injlim$


\texttt{\textbackslash intercal} is rendered as $\intercal$


\texttt{\textbackslash iota} is rendered as $\iota$


\texttt{\textbackslash jmath} is rendered as $\jmath$


\texttt{\textbackslash kappa} is rendered as $\kappa$


\texttt{\textbackslash lambda} is rendered as $\lambda$


\texttt{\textbackslash Lambda} is rendered as $\Lambda$


\texttt{\textbackslash land} is rendered as $\land$


\texttt{\textbackslash ldots} is rendered as $\ldots$


\texttt{\textbackslash leftarrow} is rendered as $\leftarrow$


\texttt{\textbackslash Leftarrow} is rendered as $\Leftarrow$


\texttt{\textbackslash leftarrowtail} is rendered as $\leftarrowtail$


\texttt{\textbackslash leftharpoondown} is rendered as $\leftharpoondown$


\texttt{\textbackslash leftharpoonup} is rendered as $\leftharpoonup$


\texttt{\textbackslash leftleftarrows} is rendered as $\leftleftarrows$


\texttt{\textbackslash leftrightarrow} is rendered as $\leftrightarrow$


\texttt{\textbackslash Leftrightarrow} is rendered as $\Leftrightarrow$


\texttt{\textbackslash leftrightarrows} is rendered as $\leftrightarrows$


\texttt{\textbackslash leftrightharpoons} is rendered as $\leftrightharpoons$


\texttt{\textbackslash leftrightsquigarrow} is rendered as $\leftrightsquigarrow$


\texttt{\textbackslash leftthreetimes} is rendered as $\leftthreetimes$


\texttt{\textbackslash leqq} is rendered as $\leqq$


\texttt{\textbackslash leqslant} is rendered as $\leqslant$


\texttt{\textbackslash lessapprox} is rendered as $\lessapprox$


\texttt{\textbackslash lessdot} is rendered as $\lessdot$


\texttt{\textbackslash lesseqgtr} is rendered as $\lesseqgtr$


\texttt{\textbackslash lesseqqgtr} is rendered as $\lesseqqgtr$


\texttt{\textbackslash lessgtr} is rendered as $\lessgtr$


\texttt{\textbackslash lesssim} is rendered as $\lesssim$


\texttt{\textbackslash ll} is rendered as $\ll$


\texttt{\textbackslash Lleftarrow} is rendered as $\Lleftarrow$


\texttt{\textbackslash lll} is rendered as $\lll$


\texttt{\textbackslash lnapprox} is rendered as $\lnapprox$


\texttt{\textbackslash lneq} is rendered as $\lneq$


\texttt{\textbackslash lneqq} is rendered as $\lneqq$


\texttt{\textbackslash lnot} is rendered as $\lnot$


\texttt{\textbackslash lnsim} is rendered as $\lnsim$


\texttt{\textbackslash longleftarrow} is rendered as $\longleftarrow$


\texttt{\textbackslash Longleftarrow} is rendered as $\Longleftarrow$


\texttt{\textbackslash longleftrightarrow} is rendered as $\longleftrightarrow$


\texttt{\textbackslash Longleftrightarrow} is rendered as $\Longleftrightarrow$


\texttt{\textbackslash longmapsto} is rendered as $\longmapsto$


\texttt{\textbackslash longrightarrow} is rendered as $\longrightarrow$


\texttt{\textbackslash Longrightarrow} is rendered as $\Longrightarrow$


\texttt{\textbackslash looparrowleft} is rendered as $\looparrowleft$


\texttt{\textbackslash looparrowright} is rendered as $\looparrowright$


\texttt{\textbackslash lor} is rendered as $\lor$


\texttt{\textbackslash lozenge} is rendered as $\lozenge$


\texttt{\textbackslash Lsh} is rendered as $\Lsh$


\texttt{\textbackslash ltimes} is rendered as $\ltimes$


\texttt{\textbackslash lVert} is rendered as $\lVert$


\texttt{\textbackslash lvertneqq} is rendered as $\lvertneqq$


\texttt{\textbackslash measuredangle} is rendered as $\measuredangle$


\texttt{\textbackslash mho} is rendered as $\mho$


\texttt{\textbackslash mid} is rendered as $\mid$


\texttt{\textbackslash models} is rendered as $\models$


\texttt{\textbackslash mp} is rendered as $\mp$


\texttt{\textbackslash mu} is rendered as $\mu$


\texttt{\textbackslash multimap} is rendered as $\multimap$


\texttt{\textbackslash nabla} is rendered as $\nabla$


\texttt{\textbackslash natural} is rendered as $\natural$


\texttt{\textbackslash ncong} is rendered as $\ncong$


\texttt{\textbackslash nearrow} is rendered as $\nearrow$


\texttt{\textbackslash neg} is rendered as $\neg$


\texttt{\textbackslash neq} is rendered as $\neq$


\texttt{\textbackslash nexists} is rendered as $\nexists$


\texttt{\textbackslash ngeq} is rendered as $\ngeq$


\texttt{\textbackslash ngeqq} is rendered as $\ngeqq$


\texttt{\textbackslash ngeqslant} is rendered as $\ngeqslant$


\texttt{\textbackslash ngtr} is rendered as $\ngtr$


\texttt{\textbackslash ni} is rendered as $\ni$


\texttt{\textbackslash nleftarrow} is rendered as $\nleftarrow$


\texttt{\textbackslash nLeftarrow} is rendered as $\nLeftarrow$


\texttt{\textbackslash nleftrightarrow} is rendered as $\nleftrightarrow$


\texttt{\textbackslash nLeftrightarrow} is rendered as $\nLeftrightarrow$


\texttt{\textbackslash nleqq} is rendered as $\nleqq$


\texttt{\textbackslash nleqslant} is rendered as $\nleqslant$


\texttt{\textbackslash nless} is rendered as $\nless$


\texttt{\textbackslash nmid} is rendered as $\nmid$


\texttt{\textbackslash not} is rendered as $\not$


\texttt{\textbackslash notin} is rendered as $\notin$


\texttt{\textbackslash nparallel} is rendered as $\nparallel$


\texttt{\textbackslash nprec} is rendered as $\nprec$


\texttt{\textbackslash npreceq} is rendered as $\npreceq$


\texttt{\textbackslash nrightarrow} is rendered as $\nrightarrow$


\texttt{\textbackslash nRightarrow} is rendered as $\nRightarrow$


\texttt{\textbackslash nshortmid} is rendered as $\nshortmid$


\texttt{\textbackslash nshortparallel} is rendered as $\nshortparallel$


\texttt{\textbackslash nsim} is rendered as $\nsim$


\texttt{\textbackslash nsubseteq} is rendered as $\nsubseteq$


\texttt{\textbackslash nsubseteqq} is rendered as $\nsubseteqq$


\texttt{\textbackslash nsucc} is rendered as $\nsucc$


\texttt{\textbackslash nsucceq} is rendered as $\nsucceq$


\texttt{\textbackslash nsupseteq} is rendered as $\nsupseteq$


\texttt{\textbackslash nsupseteqq} is rendered as $\nsupseteqq$


\texttt{\textbackslash ntriangleleft} is rendered as $\ntriangleleft$


\texttt{\textbackslash ntrianglelefteq} is rendered as $\ntrianglelefteq$


\texttt{\textbackslash ntriangleright} is rendered as $\ntriangleright$


\texttt{\textbackslash ntrianglerighteq} is rendered as $\ntrianglerighteq$


\texttt{\textbackslash nu} is rendered as $\nu$


\texttt{\textbackslash nvdash} is rendered as $\nvdash$


\texttt{\textbackslash nVdash} is rendered as $\nVdash$


\texttt{\textbackslash nvDash} is rendered as $\nvDash$


\texttt{\textbackslash nVDash} is rendered as $\nVDash$


\texttt{\textbackslash nwarrow} is rendered as $\nwarrow$


\texttt{\textbackslash odot} is rendered as $\odot$


\texttt{\textbackslash oint} is rendered as $\oint$


\texttt{\textbackslash omega} is rendered as $\omega$


\texttt{\textbackslash Omega} is rendered as $\Omega$


\texttt{\textbackslash ominus} is rendered as $\ominus$


\texttt{\textbackslash oplus} is rendered as $\oplus$


\texttt{\textbackslash oslash} is rendered as $\oslash$


\texttt{\textbackslash otimes} is rendered as $\otimes$


\texttt{\textbackslash P} is rendered as $\P$


\texttt{\textbackslash parallel} is rendered as $\parallel$


\texttt{\textbackslash partial} is rendered as $\partial$


\texttt{\textbackslash perp} is rendered as $\perp$


\texttt{\textbackslash phi} is rendered as $\phi$


\texttt{\textbackslash Phi} is rendered as $\Phi$


\texttt{\textbackslash pi} is rendered as $\pi$


\texttt{\textbackslash Pi} is rendered as $\Pi$


\texttt{\textbackslash pitchfork} is rendered as $\pitchfork$


\texttt{\textbackslash pm} is rendered as $\pm$


\texttt{\textbackslash prec} is rendered as $\prec$


\texttt{\textbackslash precapprox} is rendered as $\precapprox$


\texttt{\textbackslash preccurlyeq} is rendered as $\preccurlyeq$


\texttt{\textbackslash preceq} is rendered as $\preceq$


\texttt{\textbackslash precnapprox} is rendered as $\precnapprox$


\texttt{\textbackslash precneqq} is rendered as $\precneqq$


\texttt{\textbackslash precnsim} is rendered as $\precnsim$


\texttt{\textbackslash precsim} is rendered as $\precsim$


\texttt{\textbackslash prime} is rendered as $\prime$


\texttt{\textbackslash prod} is rendered as $\prod$


\texttt{\textbackslash projlim} is rendered as $\projlim$


\texttt{\textbackslash propto} is rendered as $\propto$


\texttt{\textbackslash psi} is rendered as $\psi$


\texttt{\textbackslash Psi} is rendered as $\Psi$


\texttt{\textbackslash qquad} is rendered as $\qquad$


\texttt{\textbackslash quad} is rendered as $\quad$


\texttt{\textbackslash Re} is rendered as $\Re$


\texttt{\textbackslash rho} is rendered as $\rho$


\texttt{\textbackslash rightarrow} is rendered as $\rightarrow$


\texttt{\textbackslash Rightarrow} is rendered as $\Rightarrow$


\texttt{\textbackslash rightarrowtail} is rendered as $\rightarrowtail$


\texttt{\textbackslash rightharpoondown} is rendered as $\rightharpoondown$


\texttt{\textbackslash rightharpoonup} is rendered as $\rightharpoonup$


\texttt{\textbackslash rightleftarrows} is rendered as $\rightleftarrows$


\texttt{\textbackslash rightrightarrows} is rendered as $\rightrightarrows$


\texttt{\textbackslash rightsquigarrow} is rendered as $\rightsquigarrow$


\texttt{\textbackslash rightthreetimes} is rendered as $\rightthreetimes$


\texttt{\textbackslash risingdotseq} is rendered as $\risingdotseq$


\texttt{\textbackslash Rrightarrow} is rendered as $\Rrightarrow$


\texttt{\textbackslash Rsh} is rendered as $\Rsh$


\texttt{\textbackslash rtimes} is rendered as $\rtimes$


\texttt{\textbackslash rVert} is rendered as $\rVert$


\texttt{\textbackslash S} is rendered as $\S$


\texttt{\textbackslash scriptscriptstyle} is rendered as $\scriptscriptstyle$


\texttt{\textbackslash scriptstyle} is rendered as $\scriptstyle$


\texttt{\textbackslash searrow} is rendered as $\searrow$


\texttt{\textbackslash setminus} is rendered as $\setminus$


\texttt{\textbackslash sharp} is rendered as $\sharp$


\texttt{\textbackslash shortmid} is rendered as $\shortmid$


\texttt{\textbackslash shortparallel} is rendered as $\shortparallel$


\texttt{\textbackslash sigma} is rendered as $\sigma$


\texttt{\textbackslash Sigma} is rendered as $\Sigma$


\texttt{\textbackslash sim} is rendered as $\sim$


\texttt{\textbackslash simeq} is rendered as $\simeq$


\texttt{\textbackslash smallfrown} is rendered as $\smallfrown$


\texttt{\textbackslash smallsetminus} is rendered as $\smallsetminus$


\texttt{\textbackslash smallsmile} is rendered as $\smallsmile$


\texttt{\textbackslash smile} is rendered as $\smile$


\texttt{\textbackslash spadesuit} is rendered as $\spadesuit$


\texttt{\textbackslash sphericalangle} is rendered as $\sphericalangle$


\texttt{\textbackslash sqcap} is rendered as $\sqcap$


\texttt{\textbackslash sqcup} is rendered as $\sqcup$


\texttt{\textbackslash sqsubset} is rendered as $\sqsubset$


\texttt{\textbackslash sqsubseteq} is rendered as $\sqsubseteq$


\texttt{\textbackslash sqsupset} is rendered as $\sqsupset$


\texttt{\textbackslash sqsupseteq} is rendered as $\sqsupseteq$


\texttt{\textbackslash square} is rendered as $\square$


\texttt{\textbackslash star} is rendered as $\star$


\texttt{\textbackslash subset} is rendered as $\subset$


\texttt{\textbackslash Subset} is rendered as $\Subset$


\texttt{\textbackslash subseteq} is rendered as $\subseteq$


\texttt{\textbackslash subseteqq} is rendered as $\subseteqq$


\texttt{\textbackslash subsetneq} is rendered as $\subsetneq$


\texttt{\textbackslash subsetneqq} is rendered as $\subsetneqq$


\texttt{\textbackslash succ} is rendered as $\succ$


\texttt{\textbackslash succapprox} is rendered as $\succapprox$


\texttt{\textbackslash succcurlyeq} is rendered as $\succcurlyeq$


\texttt{\textbackslash succeq} is rendered as $\succeq$


\texttt{\textbackslash succnapprox} is rendered as $\succnapprox$


\texttt{\textbackslash succneqq} is rendered as $\succneqq$


\texttt{\textbackslash succnsim} is rendered as $\succnsim$


\texttt{\textbackslash succsim} is rendered as $\succsim$


\texttt{\textbackslash supset} is rendered as $\supset$


\texttt{\textbackslash Supset} is rendered as $\Supset$


\texttt{\textbackslash supseteq} is rendered as $\supseteq$


\texttt{\textbackslash supseteqq} is rendered as $\supseteqq$


\texttt{\textbackslash supsetneq} is rendered as $\supsetneq$


\texttt{\textbackslash supsetneqq} is rendered as $\supsetneqq$


\texttt{\textbackslash surd} is rendered as $\surd$


\texttt{\textbackslash swarrow} is rendered as $\swarrow$


\texttt{\textbackslash tau} is rendered as $\tau$


\texttt{\textbackslash textstyle} is rendered as $\textstyle$


\texttt{\textbackslash therefore} is rendered as $\therefore$


\texttt{\textbackslash theta} is rendered as $\theta$


\texttt{\textbackslash Theta} is rendered as $\Theta$


\texttt{\textbackslash thickapprox} is rendered as $\thickapprox$


\texttt{\textbackslash thicksim} is rendered as $\thicksim$


\texttt{\textbackslash times} is rendered as $\times$


\texttt{\textbackslash top} is rendered as $\top$


\texttt{\textbackslash triangle} is rendered as $\triangle$


\texttt{\textbackslash triangledown} is rendered as $\triangledown$


\texttt{\textbackslash triangleleft} is rendered as $\triangleleft$


\texttt{\textbackslash trianglelefteq} is rendered as $\trianglelefteq$


\texttt{\textbackslash triangleq} is rendered as $\triangleq$


\texttt{\textbackslash triangleright} is rendered as $\triangleright$


\texttt{\textbackslash trianglerighteq} is rendered as $\trianglerighteq$


\texttt{\textbackslash upharpoonleft} is rendered as $\upharpoonleft$


\texttt{\textbackslash upharpoonright} is rendered as $\upharpoonright$


\texttt{\textbackslash uplus} is rendered as $\uplus$


\texttt{\textbackslash upsilon} is rendered as $\upsilon$


\texttt{\textbackslash Upsilon} is rendered as $\Upsilon$


\texttt{\textbackslash upuparrows} is rendered as $\upuparrows$


\texttt{\textbackslash varepsilon} is rendered as $\varepsilon$


\texttt{\textbackslash varinjlim} is rendered as $\varinjlim$


\texttt{\textbackslash varkappa} is rendered as $\varkappa$


\texttt{\textbackslash varliminf} is rendered as $\varliminf$


\texttt{\textbackslash varlimsup} is rendered as $\varlimsup$


\texttt{\textbackslash varnothing} is rendered as $\varnothing$


\texttt{\textbackslash varphi} is rendered as $\varphi$


\texttt{\textbackslash varpi} is rendered as $\varpi$


\texttt{\textbackslash varprojlim} is rendered as $\varprojlim$


\texttt{\textbackslash varpropto} is rendered as $\varpropto$


\texttt{\textbackslash varrho} is rendered as $\varrho$


\texttt{\textbackslash varsigma} is rendered as $\varsigma$


\texttt{\textbackslash varsubsetneq} is rendered as $\varsubsetneq$


\texttt{\textbackslash varsubsetneqq} is rendered as $\varsubsetneqq$


\texttt{\textbackslash varsupsetneq} is rendered as $\varsupsetneq$


\texttt{\textbackslash varsupsetneqq} is rendered as $\varsupsetneqq$


\texttt{\textbackslash vartheta} is rendered as $\vartheta$


\texttt{\textbackslash vartriangle} is rendered as $\vartriangle$


\texttt{\textbackslash vartriangleleft} is rendered as $\vartriangleleft$


\texttt{\textbackslash vartriangleright} is rendered as $\vartriangleright$


\texttt{\textbackslash vdash} is rendered as $\vdash$


\texttt{\textbackslash Vdash} is rendered as $\Vdash$


\texttt{\textbackslash vDash} is rendered as $\vDash$


\texttt{\textbackslash vdots} is rendered as $\vdots$


\texttt{\textbackslash vee} is rendered as $\vee$


\texttt{\textbackslash veebar} is rendered as $\veebar$


\texttt{\textbackslash vline} is rendered as $\vline$


\texttt{\textbackslash Vvdash} is rendered as $\Vvdash$


\texttt{\textbackslash wedge} is rendered as $\wedge$


\texttt{\textbackslash wp} is rendered as $\wp$


\texttt{\textbackslash wr} is rendered as $\wr$


\texttt{\textbackslash xi} is rendered as $\xi$


\texttt{\textbackslash Xi} is rendered as $\Xi$


\texttt{\textbackslash zeta} is rendered as $\zeta$



\section{Identifier variants}\label{sc.var}
The following variants are supported\footnote{Note that \texttt{\textbackslash mathcal} is not available for lowercase Latin letters.}:

\texttt{\textbackslash acute} applied on $x,X$ is rendered as $\acute{x},\acute{X}$


\texttt{\textbackslash bar} applied on $x,X$ is rendered as $\bar{x},\bar{X}$


\texttt{\textbackslash bcancel} applied on $x,X$ is rendered as $\bcancel{x},\bcancel{X}$


\texttt{\textbackslash bmod} applied on $x,X$ is rendered as $\bmod{x},\bmod{X}$


\texttt{\textbackslash boldsymbol} applied on $x,X$ is rendered as $\boldsymbol{x},\boldsymbol{X}$


\texttt{\textbackslash breve} applied on $x,X$ is rendered as $\breve{x},\breve{X}$


\texttt{\textbackslash cancel} applied on $x,X$ is rendered as $\cancel{x},\cancel{X}$


\texttt{\textbackslash check} applied on $x,X$ is rendered as $\check{x},\check{X}$


\texttt{\textbackslash ddot} applied on $x,X$ is rendered as $\ddot{x},\ddot{X}$


\texttt{\textbackslash dot} applied on $x,X$ is rendered as $\dot{x},\dot{X}$


\texttt{\textbackslash emph} applied on $x,X$ is rendered as $\emph{x},\emph{X}$


\texttt{\textbackslash grave} applied on $x,X$ is rendered as $\grave{x},\grave{X}$


\texttt{\textbackslash hat} applied on $x,X$ is rendered as $\hat{x},\hat{X}$


\texttt{\textbackslash mathbb} applied on $x,X$ is rendered as $\mathbb{x},\mathbb{X}$


\texttt{\textbackslash mathbf} applied on $x,X$ is rendered as $\mathbf{x},\mathbf{X}$


\texttt{\textbackslash mathbin} applied on $x,X$ is rendered as $\mathbin{x},\mathbin{X}$


\texttt{\textbackslash mathcal} applied on $x,X$ is rendered as $\mathcal{x},\mathcal{X}$


\texttt{\textbackslash mathclose} applied on $x,X$ is rendered as $\mathclose{x},\mathclose{X}$


\texttt{\textbackslash mathfrak} applied on $x,X$ is rendered as $\mathfrak{x},\mathfrak{X}$


\texttt{\textbackslash mathit} applied on $x,X$ is rendered as $\mathit{x},\mathit{X}$


\texttt{\textbackslash mathop} applied on $x,X$ is rendered as $\mathop{x},\mathop{X}$


\texttt{\textbackslash mathopen} applied on $x,X$ is rendered as $\mathopen{x},\mathopen{X}$


\texttt{\textbackslash mathord} applied on $x,X$ is rendered as $\mathord{x},\mathord{X}$


\texttt{\textbackslash mathpunct} applied on $x,X$ is rendered as $\mathpunct{x},\mathpunct{X}$


\texttt{\textbackslash mathrel} applied on $x,X$ is rendered as $\mathrel{x},\mathrel{X}$


\texttt{\textbackslash mathrm} applied on $x,X$ is rendered as $\mathrm{x},\mathrm{X}$


\texttt{\textbackslash mathsf} applied on $x,X$ is rendered as $\mathsf{x},\mathsf{X}$


\texttt{\textbackslash mathtt} applied on $x,X$ is rendered as $\mathtt{x},\mathtt{X}$


\texttt{\textbackslash operatorname} applied on $x,X$ is rendered as $\operatorname{x},\operatorname{X}$


\texttt{\textbackslash overleftarrow} applied on $x,X$ is rendered as $\overleftarrow{x},\overleftarrow{X}$


\texttt{\textbackslash overleftrightarrow} applied on $x,X$ is rendered as $\overleftrightarrow{x},\overleftrightarrow{X}$


\texttt{\textbackslash overline} applied on $x,X$ is rendered as $\overline{x},\overline{X}$


\texttt{\textbackslash overrightarrow} applied on $x,X$ is rendered as $\overrightarrow{x},\overrightarrow{X}$


\texttt{\textbackslash textbf} applied on $x,X$ is rendered as $\textbf{x},\textbf{X}$


\texttt{\textbackslash textit} applied on $x,X$ is rendered as $\textit{x},\textit{X}$


\texttt{\textbackslash textrm} applied on $x,X$ is rendered as $\textrm{x},\textrm{X}$


\texttt{\textbackslash textsf} applied on $x,X$ is rendered as $\textsf{x},\textsf{X}$


\texttt{\textbackslash texttt} applied on $x,X$ is rendered as $\texttt{x},\texttt{X}$


\texttt{\textbackslash tilde} applied on $x,X$ is rendered as $\tilde{x},\tilde{X}$


\texttt{\textbackslash underline} applied on $x,X$ is rendered as $\underline{x},\underline{X}$


\texttt{\textbackslash vec} applied on $x,X$ is rendered as $\vec{x},\vec{X}$


\texttt{\textbackslash widehat} applied on $x,X$ is rendered as $\widehat{x},\widehat{X}$


\texttt{\textbackslash widetilde} applied on $x,X$ is rendered as $\widetilde{x},\widetilde{X}$


\texttt{\textbackslash xcancel} applied on $x,X$ is rendered as $\xcancel{x},\xcancel{X}$


\texttt{\textbackslash xleftarrow} applied on $x,X$ is rendered as $\xleftarrow{x},\xleftarrow{X}$


\texttt{\textbackslash xrightarrow} applied on $x,X$ is rendered as $\xrightarrow{x},\xrightarrow{X}$


\texttt{\textbackslash Bbb} applied on $x,X$ is rendered as $\Bbb{x},\Bbb{X}$


\texttt{\textbackslash bold} applied on $x,X$ is rendered as $\bold{x},\bold{X}$


\end{document}
